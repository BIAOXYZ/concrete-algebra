\documentclass[ContainsChinese]{mckaybook}

\Title{Concrete Algebra}
\Subtitle{With a View Toward Abstract Algebra}
\Author{\texorpdfstring{Benjamin \scotsMc{}Kay}{Benjamin McKay}}
\Location{University College Cork}
\BibliographyFile{algebra}

\usepackage{standalone}
\usepackage{xstring}
\usepackage{amsmath}
\usepackage{mathrsfs}		% the \mathscr command for script fonts
\usepackage{mleftright}		% fixes problems with \left and \right
\usepackage{verbatim}		% For verbatim quotation of programming code.
\usepackage{asymptote}
\usepackage{siunitx}
%  longdiv.tex  v.1  (1994)  Donald Arseneau  
%
%  Work out and print integer long division problems.  Use:
%       \longdiv{numerator}{denominator}
%  The numerator and denominator (divisor and dividend) must be integers, and
%  the quotient is an integer too.  \longdiv leaves a remainder.
%  Use this in any type of TeX.

\newcount\gpten % (global) power-of-ten -- tells which digit we are doing
\countdef\rtot2 % running total -- remainder so far
\countdef\LDscratch4 % scratch

\def\longdiv#1#2{%
 \vtop{\normalbaselines \offinterlineskip
   \setbox\strutbox\hbox{\vrule height 2.1ex depth .5ex width0ex}%
   \def\showdig{$\underline{\the\LDscratch\strut}$\cr\the\rtot\strut\cr
       \noalign{\kern-.2ex}}%
   \global\rtot=#1\relax
   \count0=\rtot\divide\count0by#2\edef\quotient{\the\count0}%\show\quotient
   % make list macro out of digits in quotient:
   \def\temp##1{\ifx##1\temp\else \noexpand\dodig ##1\expandafter\temp\fi}%
   \edef\routine{\expandafter\temp\quotient\temp}%
   % process list to give power-of-ten:
   \def\dodig##1{\global\multiply\gpten by10 }\global\gpten=1 \routine
   % to display effect of one digit in quotient (zero ignored):
   \def\dodig##1{\global\divide\gpten by10
      \LDscratch =\gpten
      \multiply\LDscratch  by##1%
      \multiply\LDscratch  by#2%
      \global\advance\rtot-\LDscratch \relax
      \ifnum\LDscratch>0 \showdig \fi % must hide \cr in a macro to skip it
   }%
   \tabskip=0pt
   \halign{\hfil##\cr % \halign for entire division problem
     $\quotient$\strut\cr
     #2$\,\overline{\vphantom{\big)}%
     \hbox{\smash{\raise3.5\fontdimen8\textfont3\hbox{$\big)$}}}%
     \mkern2mu \the\rtot}$\cr\noalign{\kern-.2ex}
     \routine \cr % do each digit in quotient
}}}

\endinput % Demonstration below:

\noindent Here are some long division problems

\indent
\longdiv{12345}{13} \quad
\longdiv{123}{1234} \quad
\longdiv{31415926}{2} \quad
\longdiv{81}{3} \quad
\longdiv{1132}{99} \quad
\longdiv{86491}{94}
\bye

\usepackage{cool}
\usepackage{tikz}			% TiKZ graphics packages
\usetikzlibrary{%
arrows,
backgrounds,
calc,
decorations.pathmorphing,
fit,
intersections,
petri,
positioning,
through
}
\usepackage{pgfplots}
\pgfplotsset{compat=1.14}
\usepackage{tikz-3dplot}
\usepackage{stackengine}
\usepackage{tensor}
\usepackage{epigraph-keys}	% Handles epigraphs at the start of each chapter.
\usepackage{sagetex}
\vrefwarning
\usepackage{morewrites}
\usepackage{bookmark}% http://ctan.org/pkg/bookmark  

%%.....Mathematics Commands
\def\cprime{\('\)} 			% For Russian names

\newcommand*{\defeq}%		% for definitions, A \defeq B means A is defined to be B.
{\mathrel{\vcenter{\baselineskip0.5ex \lineskiplimit0pt
                     \hbox{\scriptsize.}\hbox{\scriptsize.}}}%
                     =}
% I don't like the default \Re and \Im for complex numbers.
\renewcommand{\Re}{\ensuremath{\operatorname{Re}}} 
\renewcommand{\Im}{\ensuremath{\operatorname{Im}}}

% Various sets
\newcommand*{\Z}[1]{\ensuremath{\mathbb{Z}^{#1}}}
\newcommand*{\N}[1]{\ensuremath{\mathbb{N}^{#1}}}
\newcommand*{\R}[1]{\ensuremath{\mathbb{R}^{#1}}}
\newcommand*{\Q}[1]{\ensuremath{\mathbb{Q}^{#1}}}
\newcommand*{\C}[1]{\ensuremath{\mathbb{C}^{#1}}}
\NewDocumentCommand\Zmod{sm}{\ensuremath{\mathbb{Z}\!/{\IfBooleanTF{#1}{\pr{#2}}{#2}}\mathbb{Z}}}

% Brackets
\newcommand*{\pr}[1]{\ensuremath{\left(#1\right)}}
\newcommand*{\curly}[1]{\ensuremath{\left\{#1\right\}}}
\newcommand*{\of}[1]{\ensuremath{\!\pr{#1}}}
\newcommand*{\equalquestion}{\stackrel{?}{=}}

% greatest common divisor
\renewcommand*{\gcd}[1]{\ensuremath{\operatorname{gcd}	\left\{{#1}\right\}}}
\newcommand*{\lcm}[1]{\ensuremath{\operatorname{lcm}\left\{{#1}\right\}}}

\usepackage{calc}

\newcommand{\longdivision}[2]{
    \settowidth{\dividendlength}{#1}
    \settowidth{\divisorlength}{#2}
    \settoheight{\dividendheight}{#1}
    \settoheight{\maxheight}{#1#2}
    \settoheight{\divisorheight}{#2}

    \begin{tikzpicture} [baseline=.5pt]
        \node at (-.5*\divisorlength-1pt,.5*\divisorheight) {#2};
        \node at (.5*\dividendlength+5pt,.5*\dividendheight) {#1};
        \draw [thick]  (0pt,-.22*\dividendheight) arc (-70:60:\maxheight*.41 and \maxheight*.82) -- ++(\dividendlength+7pt,0pt);
    \end{tikzpicture}
}

\newlength{\dividendlength}
\newlength{\divisorlength}
\newlength{\dividendheight}
\newlength{\divisorheight}
\newlength{\maxheight}

%% Two columns proofs:
% Usage:
%\begin{twocolumnproof}
%\pf{0}{0 + 0}[problem~1] \\
%\pf{a \cdot 0}{a \cdot (0 + 0)}[multiplying by \(a\)] \\
%\pf{a \cdot 0}{a \cdot 0 + a \cdot 0}[the distributive law] \\
%\pf{\text{Let } b}{-(a \cdot 0)} \\
%\pf{a \cdot 0 + b}{(a \cdot 0 + a \cdot 0) + b}[adding \(b\) to both sides] \\
%\pf{a \cdot 0 + b}{a \cdot 0 + (a \cdot 0 + b)}[the associative law for addition] \\
%\pf{0}{a \cdot 0 + 0}[the definition of \(b\)] \\
%\lastpf{0}{a \cdot 0}[the definition of \(0\)]
%\end{twocolumnproof}
\NewDocumentEnvironment{twocolumnproof}{}{\csname align*\endcsname}{\csname endalign*\endcsname}
\NewDocumentCommand\mainproofstep{mO{=}mom}{#1&#2#3\IfValueT{#4}{&{\quad}&\text{by #4}}#5}
\NewDocumentCommand\pf{mO{=}mo}{\mainproofstep{#1}[#2]{#3}[#4]{,}}
\NewDocumentCommand\lastpf{mO{=}mo}{\mainproofstep{#1}[#2]{#3}[#4]{.}}


\NewDocumentCommand\cardinality{sm}%
{%
\IfBooleanTF{#1}%
{\tensor[^{\#}]{(#2)}{}}%
{\tensor[^{\#}]{{#2}}{}}%
}%


% \congmod[p]{b}{c} means b=c (mod p).
\NewDocumentCommand\congmod{omm}{\IfValueTF{#1}{#2 \equiv #3 \pmod{#1}}{#2 \equiv #3}}

%\Proj[2]{k}
%or
%\Proj{2}
\NewDocumentCommand\Proj{om}{\IfValueTF{#1}{\mathbb{P}^{#1}\!\of{#2}}{\mathbb{P}^{#2}}}

\newcommand{\resultant}[2]{%
\ensuremath{\operatorname{res}_{#1,#2}}
}%

\newcommand{\discriminant}[1]{%
\ensuremath{\Delta_{#1}}
}%

\newcommand*{\tr}[1]{\operatorname{tr} #1}

\newcommand*{\degree}[1]{\operatorname{deg} #1}

\setstackgap{L}{.7\baselineskip}

\newcommand*{\multiplicity}[3]%\multiplicity{point}{curve}{curve}
{\ensuremath{#2#3_{#1}}}

\newcommand*{\order}[2]%\order{point}{curve}
{\ensuremath{{#2}_{#1}}}

\newcommand*{\intersectionnumber}[2]%\intersectionnumber{curve}{curve}
{\ensuremath{\deg{#1#2}}}

\NewDocumentCommand\Gal{mm}{\operatorname{Aut}{#2/#1}}

% quaternions
\newcommand*{\Quat}[1]{\ensuremath{\mathbb{H}^{#1}}}

% octonions
\newcommand*{\Oct}[1]{\ensuremath{\mathbb{O}^{#1}}}

\newcommand*{\ii}{\ensuremath{\iota}}

\NewDocumentCommand\Hessian{m}{\ensuremath{\det #1''}}
