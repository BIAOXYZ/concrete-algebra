\documentclass{amsart}
\usepackage{tikz}
\usepackage{xparse}
\colorlet{curveZero}{gray!75}
\colorlet{curveOne}{blue!60}
\colorlet{curveTwo}{brown!50!gray}
\colorlet{curveThree}{green!40!gray}
\colorlet{curveFour}{red!50!gray}
\NewDocumentCommand\DrawDotInPlot{O{}mmO{}}%
{%
\fill[gray!20,draw=gray] (axis cs:{#2},{#3}) circle (1.3pt) node[above,black,#4] {\(#1\)};%
}%
\NewDocumentCommand\DrawDot{O{}mmO{}}%
{%
\fill[gray!20,draw=gray] ({#2},{#3}) circle (1.3pt) node[above,black,#4] {\(#1\)};%
}%
\NewDocumentCommand\DrawNode{O{}m}%
{%
\fill[gray!20,draw=gray] (#2) circle (1.3pt) node[above,black] {\(#1\)};%
}%
\colorlet{axisColor}{gray!50}
\tikzstyle{shapeZero}=[fill=curveZero,opacity=.4]
\tikzstyle{shapeOne}=[fill=curveOne,opacity=.4]
\tikzstyle{shapeTwo}=[fill=curveTwo,opacity=.4]
\tikzstyle{shapeThree}=[fill=curveThree,opacity=.4]
\tikzstyle{groupElementLabel}=[minimum size=2.4em]
\tikzstyle{groupElement}=[minimum size=2.4em,shapeZero,draw=curveZero]
\tikzstyle{cosetOne}=[minimum size=2.4em,shapeOne,draw=curveOne]
\tikzstyle{cosetTwo}=[minimum size=2.4em,shapeTwo,draw=curveTwo]


\begin{document}
Given a projective plane \(P\), take a quadrilateral:
\NewDocumentCommand\dt{O{}mmO{}}{\DrawDot[#1]{#2/3}{#3/3}[#4]}
\newcommand{\lne}[4]{\draw[curveZero,very thick] ({#1/3},{#2/3}) -- ({#3/3},{#4/3});}
\newenvironment{ppdiagram}%
{%%
\begin{center}
\begin{tikzpicture}
}%%
{%%
\end{tikzpicture}
\end{center}
}%%
\begin{ppdiagram}
\dt{0}{0}
\dt{0}{3} 
\dt{3}{0}  
\dt{1}{1}
\end{ppdiagram}
Call this:
\begin{ppdiagram}
\lne{0}{3}{3}{0}
\dt{0}{0} 
\dt{0}{3} 
\dt{3}{0} 
\dt{1}{1} 
\end{ppdiagram}
the \emph{line at infinity}. Henceforth, draw it far away.
Two lines:
\begin{ppdiagram}
\lne{1.5}{0}{1.5}{3}
\lne{2.5}{0}{2.5}{3}
\dt{0}{0} 
\dt{1}{1} 
\end{ppdiagram}
are \emph{parallel} if they meet at the line at infinity.
One of the two ``finite'' points of our quadrilateral we designate the \emph{origin}.
We draw lines from it out to the two ``far away'' points: the \emph{axes}:
\begin{ppdiagram}
\lne{0}{0}{3}{0}
\lne{0}{0}{0}{3}
\dt{0}{0} 
\dt{1}{1} 
\end{ppdiagram}
Map any point \(p\) not on the line at infinity to points on the two axes:
\begin{ppdiagram}
\lne{0}{0}{3}{0}
\lne{0}{0}{0}{3}
\lne{2}{3}{2}{0}
\dt{0}{0}
\dt[p]{2}{1}[right]
\dt[p_x]{2}{0}[below]
\end{ppdiagram}
and
\begin{ppdiagram}
\lne{0}{0}{3}{0}
\lne{0}{0}{0}{3}
\lne{3}{1}{0}{1}
\dt{0}{0} 
\dt[p]{2}{1}[above]
\dt[p_y]{0}{1}[left]
\end{ppdiagram}
\end{document}