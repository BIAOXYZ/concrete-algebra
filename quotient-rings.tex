\chapter{Quotient rings}\label{chapter:quotient.rings}

Recall once again that all of our rings are assumed to be commutative rings with identity.
Suppose that \(R\) is a ring and that \(I\) is an ideal.
For any element \(r \in R\), then \emph{translate} of \(I\) by \(r\), denoted \(r+I\), is the set of all elements \(r+i\) for any \(i \in I\).
\begin{example}
If \(R=\Z{}\) and \(I=12\Z{}\) is the multiples of \(12\), then \(7+I\) is the set of integers which are 7 larger than a multiple of \(12\), i.e. the set of all numbers 
\[
\dots,7-12,7,7+12,7+2 \cdot 12, 7+3 \cdot 12, \dots
\]
which simplifies to
\[
\dots,-5,7,19,31,53, \dots
\]
But this is the same translate as \(-5+I\), since \(-15+12=7\) so \(-5+I\) is
the set of all numbers
\[
\dots,-5-12,-5,-5+12,-5+2 \cdot 12, -5+3 \cdot 12, \dots
\]
which is just 
\[
\dots,-15,-5,7,19,31,53, \dots
\]
the same sequence.
\end{example}
\begin{example}
Working again inside \(\Z{}\), \(2+2\Z{}=2\Z{}\) is the set of even integers.
\end{example}
\begin{example}
If \(R=\Q{}[x]\) and \(I=(x)\), then \(\frac{1}{2}+I\) is the set of all polynomials of the form
\[
\frac{1}{2}+xp(x)
\]
for any polynomial \(p(x)\).
\end{example}
\begin{example}
If \(i \in I\) then \(i+I=I\).
\end{example}
We add two translates by
\[
\pr{r_1+I}+\pr{r_2+I}\defeq \pr{r_1+r_2}+I,
\]
and multiply by
\[
\pr{r1+I}\pr{r_2+I}=\pr{r_1r_2}+I.
\]
\begin{lemma}\label{lemma:translates}
For any ideal \(I\) in any ring \(R\), two translates \(a+I\) and \(b+I\) are equal just when \(a\) and \(b\) differ by an element of \(I\).
\end{lemma}
\begin{proof}
We suppose that \(a+I=b+I\).
Clearly \(a\) belongs to \(a+I\), because \(0\) belongs to \(I\).
Therefore \(a\) belongs to \(a+I=b+I\), i.e. \(a=b+i\) for some \(i\) from \(I\).
\end{proof}

\begin{lemma}
The addition and multiplication operations on translates are well defined, i.e. if we find two different ways to write a translate as \(r+I\), the results of adding and multiplying translates don't depend on which choice we make of how to write them.
\end{lemma}
\begin{proof}
You write your translates as \(a_1+I\) and \(a_2+I\), and I write mine as \(b_1+I\) and \(sb_2+I\), but we suppose that they are the same subsets of the same ring: \(a_1+I=b_1+I\) and \(a_2+I=b_2+I\).
By lemma~\vref{lemma:translates}, \(a_1=b_1+i_i\) and \(a_2=b_2+i_2\) for some elements \(i_1, i_2\) of \(I\). 
So then
\begin{align*}
a_1+a_2+I
&=
b_1+i_1+b_2+i_2+I,
\\
&=
b_1+b_2+\pr{i_1+i_2}+I,
\\
&=
b_1+b_2+I.
\end{align*}
The same for multiplication.
\end{proof}

If \(I\) is an ideal in a commutative ring \(R\), the \emph{quotient ring}\define{quotient ring} \(R/I\)\Notation{R/I}{R/I}{quotient of a ring by an ideal} is the set of all translates of \(I\), with addition and multiplication of translates as above.
The reader can easily prove:
\begin{lemma}
For any ideal \(I\) in any commutative ring \(R\), \(R/I\) is a commutative ring.
If \(R\) has an identity element then \(R/I\) has an identity element.
\end{lemma}

\begin{example}
If \(R\defeq\Z{}\) and \(I\defeq m\Z{}\) for some integer \(m\) then \(R/I=\Zmod{m}\) is the usual ring of remainders modulo \(m\).
\end{example}
\begin{example}
If \(p(x,y)\) is an irreducible nonconstant polynomial over a field \(k\) and \(I=(p(x,y))=p(x,y)k[x]\) is the associated ideal, then \(k[x,y]/I=k[X]\) is the  ring of regular functions on the algebraic plane curve \(X=(p(x,y)=0)\).
\end{example}

\begin{lemma}
A ideal \(I\) in a ring \(R\) with identity is all of \(R\) just when \(1\) is in \(I\).
\end{lemma}
\begin{proof}
If \(1\) lies in \(I\), then any \(r\) in \(R\) is \(r=r1\) so lies in \(I\), so \(I=R\).
\end{proof}

An ideal \(I\) in a ring \(R\) is \emph{prime}\define{prime ideal}\define{ideal!prime} if \(I\) is not all of \(R\) and, for any two elements \(a, b\) of \(R\), if \(ab\) is in \(I\) then either \(a\) or \(b\) lies in \(I\).

\begin{example}
The ideal \(p\Z{}\) in \(\Z{}\) generated by any prime number is prime.
\end{example}
\begin{example}
The ring \(R\) is \emph{not} prime in \(R\), because the definition explicitly excludes it.
\end{example}
\begin{example}
Inside \(R\defeq \Z{}[x]\), the ideal \((x)\) is prime, because in order to have a factor of \(x\) sitting in a product, it must lie in one of the factors.
\end{example}
\begin{example}
Inside \(R\defeq \Z{}[x]\), the ideal \((2x)\) is \emph{not} prime, because \(a=2\) and \(b=x\) multiply to a multiple of \(2x\), but neither factor is a multiple of \(2x\).
\end{example}
\begin{example}
If \(p(x)\) is an irreducible polynomial, then \((p(x))\) is a prime ideal.
\end{example}
\begin{example}
The regular functions on an algebraic curve \(p(x,y)=0\) over a field \(k\) constitute the ring \(R/I\) where \(R=k[x,y]\) and \(I=(p(x,y))\).
The curve is irreducible just when the ideal is prime.
\end{example}
\begin{example}
Take a field \(k\).
In the quotient ring \(R=k[x,y,z]/(z^2-xy)\), the ideal \((z)\) contains \(z^2=xy\), but does not contain \(x\) or \(y\), so is not prime.
\end{example}

\begin{lemma}
An ideal \(I\) in a commutative ring with identity \(R\) is prime just when \(R/I\) has no zero divisors, i.e. no nonzero elements \(\alpha, \beta\) can have \(\alpha\beta=0\).
\end{lemma}
\begin{proof}
Write \(\alpha\) and \(\beta\) as translates \(\alpha=a+I\) and \(\beta=b+I\).
Then \(\alpha\beta=0\) just when \(ab+I=I\), i.e. just when \(ab\) lies in \(I\).
On the other hand, \(\alpha=0\) just when \(a+I=I\), just when \(a\) is in \(I\).
\end{proof}

A ideal \(I\) in a ring \(R\) is \emph{maximal}\define{ideal!maximal}\define{maximal ideal} if \(I\) is not all of \(R\) but any ideal \(J\) which contains \(I\) is either equal to \(I\) or equal to \(R\) (nothing fits in between).

\begin{lemma}
If \(I\) is an ideal in a commutative ring \(R\) with identity, then \(I\) is a maximal ideal just when \(R/I\) is a field.
\end{lemma}
\begin{proof}
Suppose that \(I\) is a maximal ideal.
Take any element \(\alpha \ne 0\) of \(R/I\) and write it as a translate \(\alpha=a+I\).
Since \(\alpha \ne 0\), \(a\) is not in \(I\).
We know that \((a)+I=R\), since \(I\) is maximal.
But then \(1\) lies in \((a)+I\), so \(1=ab+i\) for some \(b\) in \(R\) and \(i\) in \(I\).
Expand out to see that 
\[
\frac{1}{\alpha}=b+I.
\]
So nonzero elements of \(R/I\) have reciprocals.

Suppose instead that \(R/I\) is a field.
Take an element \(a\) in \(R\) not in \(I\).
Then \(a+I\) has a reciprocal, say \(b+I\).
But then \(ab+I=1+I\), so that \(ab=1+i\) for some \(i\) in \(I\).
So \((a)+I\) contains \((ab)+I=(1)=R\). 
\end{proof}

\begin{example}
If \(p\) is a prime number, then \(I \defeq p\Z{}\) is maximal in \(R\defeq \Z{}\), because any other ideal \(J\) containing \(I\) and containing some other  integer, say \(n\), not a multiple of \(p\), must contain the greatest common divisor of \(n\) and \(p\), which is \(1\) since \(p\) is prime.
We recover the fact that \(\Zmod{p}\) is a field.
\end{example}
\begin{example}
The ideals in \(R\defeq \Zmod{4}\) are \((0),(1)=\Zmod{4}\) and \((2)\cong \Zmod{2}\), and the last of these is maximal.
\end{example}
\begin{example}
Take a field \(k\) and let \(R\defeq k[x]\).
Every ideal \(I\) in \(R\) has the form \(I=(p(x))\), because each ideal is generated by the greatest common divisor of its elements.
We have seen that \(R/I\) is prime just when \(p(x)\) is irreducible.
To further have \(R/I\) a field, we need \(I=(p(x))\) to be maximal.
Take some polynomial \(q(x)\) not belonging to \(I\), so not divisible by \(p(x)\).
Since \(p(x)\) is irreducible, \(p(x)\) has no factors, so \(\gcd{p(x),q(x)}=1\) in \(k[x]\), so that the ideal generated by \(p(x),q(x)\) is all of \(R\).
Therefore \(I=(p(x))\) is maximal, and the quotient \(R/I=k[x]/(p(x))\) is a field.
\end{example}
\begin{example}
If \(k\) is an algebraically closed field and \(R\defeq k[x]\) then every irreducible polynomial is linear, so every maximal ideal in \(R\) is \(I=(x-c)\) for some constant \(c\).
\end{example}


\begin{problem}{quotient.rings:ideals.in.Zm}
For any positive integer \(m \ge 2\), what are the prime ideals in \(\Zmod{m}\), and what are the maximal ideals?
\end{problem}

\section{Sage}

We take the ring \(R=\Q{}[x,y]\) and then quotient out by the ideal \(I=(x^2y,xy^3)\) to produce the ring \(S=R/I\).
The generators \(x,y\) in \(R\) give rise to elements \(x+I,y+I\) in \(S\), which are renamed to \(a,b\) for notational convenience.
\begin{sageblock}
R.<x,y> = PolynomialRing(QQ)
I = R.ideal([x^2*y,x*y^3])
S.<a,b> = R.quotient_ring(I)
(a+b)^4
\end{sageblock}
which yields \(\sage{(a+b)^4}\).

We can define \(R=\Q{}[x,y]\), \(I=(y^2)\), \(S=R/I\) where we label \(x+I, y+I\) as \(a,b\), \(T=\Q{}[z]\) and define a morphism \(\phi \colon S \to T\) by  \(\phi(a)=z^3, \phi(b)=0\):
\begin{sageblock}
R.<x,y> = PolynomialRing(QQ)
I = R.ideal([y^2])
S.<a,b> = R.quotient_ring(I)
T.<z> = PolynomialRing(QQ)
phi = S.hom([z^3, 0],T)
phi(a^2+b^2)
\end{sageblock}
yielding \(\sage{phi(a^2+b^2)}\).


\section{Automorphisms of splitting fields}
\begin{theorem}\label{theorem:splitting.fields.exist}
Suppose that \(k\) is a field and \(p(x)\) is a nonconstant polynomial over \(k\), say of degree \(n\).
Any two splitting fields for \(p(x)\) over \(k\) are isomorphic by an isomorphism which is the identity on \(k\).
\end{theorem}
\begin{proof}
We already know that splitting fields exist, by adding roots to irreducible factors of \(p(x)\).
Each time we add a root, the dimension of the extension field over \(k\) is the degree of the irreducible factor.
We then split off at least one linear factor in the extension field, and repeat.
So at most \(n!\) degree in total.

Given any splitting field \(K\) for an irreducible polynomial \(p(x)\), pick any root \(\alpha \in K\) of \(p(x)\) and map
\[
f(x) \in k[x] \mapsto f(\alpha) \in K,
\]
to see that \(K \cong k[x]/I\).
Hence \(K\) is uniquely determined up to isomorphism.

Suppose instead that \(p(x)\) is reducible, and that \(K,L\) are splitting fields of \(p(x)\).
Pick an irreducible factor \(q(x)\) of \(p(x)\).
Then the subfields of \(K,L\) generated by adding a root of \(q(x)\) to \(k\) are isomorphic, since \(q(x)\) is irreducible, so we can assume that these subfields are the same field.
We need only split \(p(x)\) over that field, or quotient out all copies of \(q(x)\) from \(p(x)\) and split the result.
Apply induction on the degree of \(p(x)\).
\end{proof}
\begin{example}
Adding a root to \(x^2+1\) over \(\R{}\) yields the splitting field \(\C{}\).
\end{example}
\begin{theorem}\label{theorem:irreducible.orbit}
Suppose that \(K\) is a splitting field of an irreducible polynomial \(p(x)\) over a field \(k\).
If \(\alpha,\beta \in K\) are roots of \(p(x)\) then the Galois group of \(K\) over \(k\) has an element which takes \(\alpha\) to \(\beta\).
In particular, \(K\) is a Galois extension of \(k\).
\end{theorem}
\begin{proof}
As in the proof of theorem~\vref{theorem:splitting.fields.exist}, there is a field isomorphism \(k[x]/I \to K\) taking \(x \mapsto \alpha\), and another field isomorphism \(k[x]/I \to K\) taking \(x \mapsto \beta\).
\end{proof}

\section{Transcendental numbers}
A number \(\alpha \in \C{}\) is \emph{algebraic} if it is the solution of a polynomial equation \(p(\alpha)=0\) where \(p(x)\) is a nonzero polynomial with rational coefficients.
A number which is not algebraic is called \emph{transcendental}.
More generally, given a field extension \(K\) of a field \(k\), an element \(\alpha \in K\) is \emph{algebraic} over \(k\) if it is a root of a nonzero polynomial \(p(x)\) with coefficients in \(k\).
\begin{theorem}
For any field extension \(K\) of a field \(k\), an element \(\alpha \in K\) is transcendental if and only if the field 
\[
k(\alpha) = \Set{\frac{p(\alpha)}{q(\alpha)}|\frac{p(x)}{q(x)} \in k(x) \text{ and } q(\alpha) \ne 0}
\]
is isomorphic to \(k(x)\).
\end{theorem}
\begin{proof}
If \(\alpha\) is transcendental, then the map
\[
f \colon \frac{p(x)}{q(x)} \in k(x) \mapsto \frac{p(\alpha)}{q(\alpha)} \in k(\alpha)
\]
is clearly well defined, onto, and preserves all arithmetic operations.
To show that \(f\) is 1-1 is the same as showing that \(f\) has trivial kernel.
Suppose that \(p(x)/q(x)\) lies in the kernel of \(f\).
Then \(p(\alpha)=0\), so \(p(x)=0\), so \(p(x)/q(x)=0\).
So the kernel is trivial, and so \(f\) is a bijection preserving all arithmetic operations, so \(f\) is an isomorphism of fields.

On the other hand, take a element \(\alpha\) in \(K\) and suppose that there is some isomorphism of fields
\[
g \colon k(x) \to k(\alpha).
\]
Let \(\beta\defeq g(x)\).
Because \(g\) is a field isomorphism, all arithmetic operations carried out on \(x\) must then be matched up with arithmetic operations carried out on \(\beta\), so
\[
g\left(\frac{p(x)}{q(x)}\right)=\frac{p(\beta)}{q(\beta)}.
\]
Because \(g\) is an isomorphism, some element must map to \(\alpha\), say
\[
g\left(\frac{p_0(x)}{q_0(x)}\right)=\alpha.
\]
So
\[
\frac{p_0(\beta)}{q_0(\beta)}=\alpha.
\]
So \(k(\beta)=k(\alpha)\).
Any algebraic relation on \(\alpha\) clearly gives one on \(\beta\) and vice versa.
Therefore \(\alpha\) is algebraic if and only if \(\beta\) is.
Suppose that \(\beta\) is algebraic.
Then \(q(\beta)=0\) for some polynomial \(q(x)\), and then \(g\) is not defined on \(1/q(x)\), a contradiction.
\end{proof}

\section{Finite fields}
We are going to make a complete list of all finite fields.
Pick a prime number \(p\) and work over the field \(k=\Zmod{p}\).
For any integer \(n>0\), the polynomial \(x^{p^n}-x\) has two obvious linear factors: \(x\) and \(x-1\), and then factors as
\[
x^{p^n}-x
=
x(x-1)(x^{-2+p^n} + x^{-3+p^n} + \dots + x + 1).
\]
\begin{lemma}\label{lemma:splits.neatly}
For any prime \(p\), over any extension field \(K\) of the field \(k=\Zmod{p}\), the polynomial 
\[
x^{p^n}-x
\]
has no multiple roots.
\end{lemma}
\begin{proof}
Let \(c(x)\defeq x^{p^n}-x\).
As above, write \(x(x-1)b(x)=c(x)\).
Note that \(b(0)=1\ne 0\) and \(b(1)=p^n-1=-1\ne 0\).
If, in some field \(K\), we can factor out a multiple root from \(b(x)\), that multiple root is not at \(x=0\) or \(x=1\), so we can factor out the same multiple root from \(c(x)=x(x-1)b(x)\).
Take the derivative of \(c(x)\):
\[
c'(x)= \frac{d}{dx} \pr{x^{p^n}-x}=p^n x^{-1+p^n} - 1= -1,
\]
since \(p=0\) in \(k\).
But then \(c'(x)\) has no roots in any field extension \(K\) of \(k\), so \(c(x)\) has no multiple roots over \(K\), so \(b(x)\) has no multiple roots over \(K\).
\end{proof}
\begin{lemma}
For any finite field \(k\) of characteristic \(p\), the \emph{Frobenius morphism}\define{Frobenius morphism}\define{morphism!Frobenius} \(f(x)=x^p\) is an automorphism of fields \(f \colon k \to k\), i.e. is a bijection preserving addition, subtraction, multiplication and division, taking \(0 \mapsto 0\) and \(1 \mapsto 1\).
\end{lemma}
\begin{proof}
Clearly \(0^p=0\) and \(1^p=1\).
For any \(b,c \in k\), clearly \((bc)^p)=b^pc^p\).
The binomial theorem gives
\[
(b+c)^p = b^p + p b^{p-1} c + \frac{p(p-1)}{2} b^{p-2} c^2 + \dots + \frac{p!}{j!(p-j)!} b^{p-j} c^j + \dots + c.
\]
Every term except \(b\) and \(c\) has a factor of \(p\) in it, and \(p=0\) in \(k\), so
\[
(b+c)^p=b^p+c^p.
\]
The same for \((b-c)^p=b^p-c^p\). 
If \(f(b)=0\), then \(b^p=0\) so \(b \cdot b^{p-1}=0\) so \(b=0\) or \(b^{p-1}=0\), and by induction, we find that \(b=0\).
If \(f(b)=f(c)\) then \(f(b-c)=0\) so \(b-c=0\) so \(b=c\), i.e. \(f\) is 1-1.
But then, since \(k\) is finite, \(f\) is a bijection.
\end{proof}
\begin{theorem}
For any prime \(p\), the splitting field \(K\) of the polynomial
\[
c(x) = x^{p^n}-x
\]
over the field \(k=\Zmod{p}\) has \(p^n\) elements, every one of which is a root of \(c(x)\).
Every finite field \(K\) is obtained uniquely as the splitting field of \(c(x)\) for some prime number \(p\) and integer \(n \ge 1\), and so is uniquely determined up to isomorphism by its number of elements \(p^n\).
\end{theorem}
\begin{proof}
There are \(p^n\) roots of \(c(x)=x^{p^n}-x\) over its splitting field \(K\), since \(b(x)\) splits into linear factors, and, by lemma~\vref{lemma:splits.neatly} each linear factor gives a distinct root.
Given any roots \(\alpha,\beta\) of \(c(x)\), 
\[
\alpha^{p^n}=\alpha, \ \beta^{p^n}=\beta.
\]
Apply the Frobenius morphism:
\[
(\alpha+\beta)^{p^n} = \alpha^{p^n}+\beta^{p^n} = \alpha + \beta,
\]
so \(c(\alpha+\beta)=0\).
Similarly 
\[
(\alpha\beta)^{p^n}=\alpha^{p^n}\beta^{p^n}=\alpha\beta,
\]
so \(c(\alpha\beta)=0\).
Hence the roots of \(c(x)\) form a field, over which \(c(x)\) splits, and so \(b(x)\) splits.
So every element of \(K\) is a root of \(c(x)\), and so \(K\) is a splitting field of \(c(x)\) with \(p^n\) elements.
By the existence and uniqueness of splitting fields in theorem~\vref{theorem:splitting.fields.exist}, \(K\) is the unique splitting field of \(c(x)\).

Take any finite field \(L\), say of characteristic \(p\).
We map \(0,1,2,\dots,p-1\) in \(k=\Zmod{p}\) to \(0,1,2,\dots,p-1\) in \(L\), a morphism of fields.
Hence \(L\) is a vector space over \(k\), of finite dimension, say \(n\), and so has \(p^n\) elements.
The Frobenius morphism is an invertible \(k\)-linear map of \(L\), generating a subgroup of the group of invertible \(k\)-linear maps \(L \to L\).
This subgroup is finite, since there are only finitely many elements of \(L\), so finitely many permutations of those elements.
This subgroup is generated by a single element, so is cyclic.
By theorem~\vref{theorem:cyclic.groups}, this subgroup is isomorphic to \(\Zmod{\ell}\), for some integer \(\ell > 0\), so every element of \(L\) satisfies some equation \(x^{p^{\ell}}=x\).
\end{proof}
For any finite field \(k\), say with \(p^n\) elements, we can associate to any \(\alpha,\beta \in k\) the quadratic equation \((x-\alpha)(x-\beta)=0\) which has roots \(\alpha,\beta\).
On the other hand, given a quadratic equation \(x^2+bx+c=0\), we can ask whether it has roots.
Those with roots are given as \(x^2+bx+c=(x-\alpha)(x-\beta)\) for some \(\alpha,\beta\), unique up to swapping the order in which we write them down.
As there are \(p^n\) elements, there are 
\[
\frac{p^n(p^n+1)}{2}
\]
pairs \(\alpha,\beta\) of possible roots, up to swapping order.
But there are \(p^{2n}\) quadratic equations \(x^2+bx+c\): choose any \(b,c\) from \(k\).
So 
\[
p^{2n}-\frac{p^n(p^n+1)}{2}
=
\frac{p^n(p^n-1)}{2}
\]
quadratic equations over \(k\) have no solution in \(k\).

In particular, let \(k_1 \subset k_2 \subset k_3 \dots\) be finite fields or orders \(2,4,8\) and so on.
Each \(k_{n+1}\) is the splitting field of the polynomial \(z^2+z+\alpha\) over \(k_n\) where \(\alpha\) is any element of \(k_n\) not belonging to \(k_{n-1}\). 
In particular, we can see which quadratic equations have solutions in which fields of characteristic 2.
Note that we can't use the quadratic formula in fields of characteristic 2, because the quadratic formula has a 2 in it.

\begin{problem}{fields:double}
Prove by induction that if \(k\) is a finite field of characteristic 2, then every element \(\alpha\) of \(k\) is a square, i.e. \(\alpha=\beta^2\) for some \(\beta\).
\end{problem}
\begin{answer}{fields:double}
Let \(k_1 \subset k_2 \subset k_3 \dots\) be finite fields or orders \(2,4,8\) and so on.
Our result is clear if \(k=k_1=\Zmod{2}\).
By induction, suppose that our result is true for \(k_1, k_2, \dots, k_{n-1}\).
All of the elements \(\alpha\) of \(k_n\) that are not elements of \(k_{n-1}\) satisfy \(\alpha^2+\alpha+c=0\) for some \(c\) in \(k_{n-1}\), as we saw above.
By induction, \(c=b^2\) for some element \(b\) of \(k_{n-1}\).
So \(\alpha^2+\alpha+b^2=0\), i.e. \(\alpha^2+b^2=\alpha\).
Expand out \((\alpha+b)^2\) to find \((\alpha+b)^2=\alpha\).
\end{answer}

\section{Sage}
Lets get sage to construct the splitting field \(K\) of \(p(x)\defeq x^2+x+1\) over the field \(k=\Zmod{2}\).
Sage constructs splitting fields by first building a ring, which we will call \(R\), of polynomials over a field.
In our case, we build \(R=k[x]\).
Then we let \(K\) be the field generated by an element \(a\), which corresponds to the variable \(x\) in the polynomial ring \(R\).
\begin{sageblock}
R.<x> = PolynomialRing(GF(2))
K.<a> = (x^2 + x + 1).splitting_field()
\end{sageblock}
We can then make a list of some elements of \(K\), call it \(S\), and compute out addition and multiplication tables:
\begin{sageblock}
S=[0,1,a,a+1]
print("Addition table:")
for i in range(0,4):
    for j in range(0,4):
        print("({})+({})={}".format(S[i],S[j],S[i]+S[j]))
print("Multiplication table:")
for i in range(0,4):
    for j in range(0,4):
        print("({})*({})={}".format(S[i],S[j],S[i]*S[j]))
\end{sageblock}


